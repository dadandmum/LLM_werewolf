
\section{Discussion}

\subsection{Application of Humanized Agents}

The Humanized Agent possesses broad application potential beyond the werewolf-like social game demonstrated in this work, particularly in areas that could benefit from modeling human behavior. For example, in the field of psychology, the introduction of personality types into LLM agents has created unprecedented opportunities for analyzing and generating language data on a massive scale \citep{demszky2023using}. Meanwhile, in sociology, the diversity of intelligent agents can aid in the establishment of social prototypes \citep{park2022social}. Additionally, diverse intelligent agents are also significant for the construction of Virtual Reality environments \citep{wan2024building}.


\subsection{Future Work and Limitations}

In this work, we have implemented the initial framework and subsystems of the Humanized Agent and conducted preliminary evaluations. Future experiments can build upon this system by introducing additional subsystems to increase its richness. Furthermore, the stability of the system is also an issue worth investigating. How to make the system more efficiently and stably generate high-quality content without causing systemic crashes while reducing the hard coding of related prompts is something that warrants further research. Finally, during the course of studying this issue, we observed that the gaming system based on LLM agents exhibited diversity distinct from traditional strategic models, which might be an inspiration for the study of social gaming problems.

Although the Humanized Agent has shown great potential in the production of non-repetitive content, the means to measure the richness of its content are still lacking. Although this paper proposes three numerical testing methods, the stability of their tests still deserves more evaluation, and they fall short of standardizing the measurement of textual content richness. Future research may need to explore more in terms of the stability of LLM assessments.

\section{Conclusion}

In this paper, we have designed a social gaming framework based on LLM (Large Language Model) agents named Humanized Agent. This framework integrates three subsystems to achieve diversified descriptions of agent personalities, dynamic preferences between agents, and strategic reasoning analysis based on the current game state. The implementation of these features allows LLM agents based on this framework to generate richer and more varied interactive content.

Additionally, we propose three tools for detecting the diversity of content generated by LLM agents, and we tested our framework in an experimental environment using a variant of the game "One Night Ultimate Werewolf." Test results indicate that our composite framework is capable of producing richer interactive content. For future work, we aim to enhance our model to provide more robust tools for generating interactive content.
